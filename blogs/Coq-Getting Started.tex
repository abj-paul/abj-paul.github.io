% Created 2025-03-22 Sat 13:14
% Intended LaTeX compiler: pdflatex
\documentclass[11pt]{article}
\usepackage[utf8]{inputenc}
\usepackage[T1]{fontenc}
\usepackage{graphicx}
\usepackage{longtable}
\usepackage{wrapfig}
\usepackage{rotating}
\usepackage[normalem]{ulem}
\usepackage{amsmath}
\usepackage{amssymb}
\usepackage{capt-of}
\usepackage{hyperref}
\author{Abhijit Paul}
\date{\textit{<2025-03-22 Sat>}}
\title{Getting Started with Coq}
\hypersetup{
 pdfauthor={Abhijit Paul},
 pdftitle={Getting Started with Coq},
 pdfkeywords={},
 pdfsubject={},
 pdfcreator={Emacs 29.3 (Org mode 9.6.15)}, 
 pdflang={English}}
\begin{document}

\maketitle
\tableofcontents

Coq is a proof assistant. What it means is - writing entire proofs on our own is tedious and error-prone. So we can get computer to help is in writing proofs! We are considering computer science now. Can we prove anything about computer?

Axioms
Proofs
Formal System


Paradox: This statement is not true.

If its true, that means the statement is true.
If its false, then the statement is true but i said the statement is false.

Paradox: This statement is not true --> Provable. (The undefiniability theorem, "Arithmetical truth can't be defined in arithmetic".

Goedl encoding. For decoding, prime number factorization


Statement A => God (Statement A) = s
Proof of A => God (Proof A) = p

Proof (p,s) = P is a proof of S.
Pr(s) = \(\E p Proof(p,s)\)  
\end{document}
